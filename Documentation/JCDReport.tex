\documentclass[a4paper,12pt]{article}


% add more packages if necessary
\usepackage{xspace}
%\usepackage{graphicx}
%\usepackage{xcolor}
%\usepackage{hyperref}


% TODO: Add your group name
\newcommand{\groupname}{jcd\_dumplings\xspace}


\title{
Project Report \\ 
Group \groupname \\
\vspace{5mm}
\large Java and C\# in depth, Spring 2013
}
\author{
% TODO: Add your names here
Elmer Lukas \\
Ma Hua \\
Nussbaumer Ivo
}
\date{\today}



\begin{document}
\maketitle

\section{Introduction}

This document describes the design and implementation of the \emph{Personal Virtual File System} of group \emph{\groupname}. The project is part of the course \emph{Java and C\# in depth} at ETH Zurich. The following sections describe each project phase, listing the requirements that were implemented and the design decisions taken. The last section describes a use case of using the \emph{Personal Virtual File System}.

% PART I: VFS CORE
% --------------------------------------

\section{VFS Core}

% TODO: Remove this line
\textbf{[This section has to be completed by April 8th.]}

%TODO: Remove this text and replace it with actual content
\emph{Give a short description (1-2 paragraphs) of what VFS Core is.}


\subsection{Requirements}

% TODO: Remove this text and replace it with actual content
\emph{Describe which requirements (and possibly bonus requirements) you have implemented in this part. Give a quick description (1-2 sentences) of each requirement. List the software elements (classes and or functions) that are mainly involved in implementing each requirement.}


\subsection{Design}

% TODO: Remove this text and replace it with actual content
\emph{Give an overview of the design of this part and describe in general terms how the implementation works. You can mention design patterns used, class diagrams, definition of custom file formats, network protocols, or anything else that helps understand the implementation.}



% PART II: VFS Browser
% --------------------------------------

\section{VFS Browser}

% TODO: Remove this line
\textbf{[This section has to be completed by April 22nd.]}

%TODO: Remove this text and replace it with actual content
\emph{Give a short (1-2 paragraphs) description of what VFS Browser is.}


\subsection{Requirements}

% TODO: Remove this text and replace it with actual content
\emph{Describe which requirements (and possibly bonus requirements) you have implemented in this part. Give a quick description (1-2 sentences) of each requirement. List the software elements (classes and or functions) that are mainly involved in implementing each requirement.}


\subsection{Design}

% TODO: Remove this text and replace it with actual content
\emph{Give an overview of the design of this part and describe in general terms how the implementation works. You can mention design patterns used, class diagrams, definition of custom file formats, network protocols, or anything else that helps understand the implementation.}


\subsection{Integration}

% TODO: Remove this text and replace it with actual content
\emph{If you had to change the design or API of the previous part, describe the changes and the reasons for each change here.}



% PART III: Synchronization Server
% --------------------------------------

\section{Synchronization Server}

% TODO: Remove this line
\textbf{[This section has to be completed by May 13th.]}

%TODO: Remove this text and replace it with actual content
\emph{Give a short (1-2 paragraphs) description of what VFS Browser is.}


\subsection{Requirements}

% TODO: Remove this text and replace it with actual content
\emph{Describe which requirements (and possibly bonus requirements) you have implemented in this part. Give a quick description (1-2 sentences) of each requirement. List the software elements (classes and or functions) that are mainly involved in implementing each requirement.}


\subsection{Design}

% TODO: Remove this text and replace it with actual content
\emph{Give an overview of the design of this part and describe in general terms how the implementation works. You can mention design patterns used, class diagrams, definition of custom file formats, network protocols, or anything else that helps understand the implementation.}


\subsection{Integration}

% TODO: Remove this text and replace it with actual content
\emph{If you had to change the design or API of the previous part, describe the changes and the reasons for each change here.}



% PART IV: Quick Start Guide
% --------------------------------------

\section{Quick Start Guide}

% TODO: Remove this line
\textbf{[optional: This part has to be completed by April 8th.]}

% TODO: Remove this text and replace it with actual content
\emph{If you have a command line interface for your VFS, describe here the commands available (e.g. ls, copy, import).} \\ \\ \\


% TODO: Remove this line
\noindent\textbf{[This part has to be completed by May 13th.]}

% TODO: Remove this text and replace it with actual content
\emph{Describe how to realize the following use case with your system. Describe the steps involved and how to perform each action (e.g. command line executions and arguments, menu entries, keyboard shortcuts, screenshots). The use case is the following:
\begin{enumerate}
\item Start synchronization server on localhost.
\item Create account on synchronization server.
\item Create two VFS disks (on the same machine) and link them to the new account.
\item Import a directory (recursively) from the host file system into Disk 1.
\item Dispose Disk 1 after the synchronization finished.
\item Export the directory (recursively) from Disk 2 into the host file system.
\item Stop synchronization server.
\end{enumerate}
}


\end{document}
